\documentclass{jsarticle}

\title{2.準備運動}

\date{\today}

\author{奥 屋 直 己}

\maketitle

\section{コンパイル・リンク}
\subsection{コンパイル}
コンパイルとは、プログラミング言語で書かれたソースプログラムを、コンピュータで実行可能な実行形式に変換することである。その際、ソースプログラムを実行形式にいきなり変換するわけではなく、プリプロセッサという前処理を行う。C言語でよく使われるものの例として#define命令と#include命令がある。#include命令では#includeが書かれた行に指定されたファイルの中身をそのままその行に挿入する。例えば#include<stdio.h>とすると、stdio.hというファイルの中身が#include<stdio.h>と書いてある部分と置き換わる。この#include命令で読み込まれる拡張子が.hのファイルをヘッダファイルという。
\subsection{リンク}
だんだんプログラムが複雑になっていくとソースプログラムが長くり、1つのソースプログラムでは全体が見づらくなる。また、一部を修正しただけでソースコード全体をコンパイルしなければならなくなり、コンパイルに時間がかかるようになる。これらの問題を解消するために、プログラムをいくつかのファイルに分けてそれぞれをコンパイルし、それらを結合する、分割コンパイルという事を行う。その際、各ソースコードをコンパイルしてできたファイルをオブジェクトファイル。このオブジェクトファイル同士を連結し、実行形式とすることをリンクという。

\section{変数とポインタ}
